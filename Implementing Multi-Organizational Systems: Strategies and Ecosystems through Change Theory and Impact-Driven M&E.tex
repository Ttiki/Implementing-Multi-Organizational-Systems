\documentclass[conference]{IEEEtran}
\IEEEoverridecommandlockouts
% The preceding line is only needed to identify funding in the first footnote. If that is unneeded, please comment it out.
\usepackage{cite}
\usepackage{amsmath,amssymb,amsfonts}
\usepackage{algorithmic}
\usepackage{graphicx}
\usepackage{textcomp}
\usepackage{xcolor}
\def\BibTeX{{\rm B\kern-.05em{\sc i\kern-.025em b}\kern-.08em
    T\kern-.1667em\lower.7ex\hbox{E}\kern-.125emX}}
\begin{document}

\title{Implementing Multi-Organizational Systems: Strategies and Ecosystems through Change Theory and Impact-Driven M\&E\\
{\footnotesize \textsuperscript{}}
\thanks{}
}

\author{\IEEEauthorblockN{Clément Combier}
\IEEEauthorblockA{\textit{Master 2 SIGLIS} \\
\textit{Université de Pau et des Pays de l'Adour}\\
Anglet, France \\
clement.combier@etud.univ-pau.fr}}
\maketitle

\section*{Abbreviation}
\begin{itemize}
    \item[] M\&E : Monitoring and Evaluation system 
    \item[] ToC : Theory of Change 
\end{itemize}

\begin{abstract}
This paper delves into the development of an impact-oriented Monitoring and Evaluation (M\&E) methodology within multi-organizational systems, inspired by the UNITA project – an innovative alliance of twelve European universities. Drawing on the principles outlined in Methods Lab's 'When and How to Develop an Impact-Oriented M\&E system' (2016), this study aims to rethink traditional M\&E approaches, emphasizing long-term impact and sustainable change.

We will use Theory of Change, which provide a structured framework to articulate and assess expected outcomes in complex collaborative environments. This approach is specifically interesting to our context of how inter-university collaborations can be optimized for greater effectiveness and increased impact.

Through practical examples and case studies, including those from the UNITA initiative, this paper illustrates the application of the Theory of Change in conjunction with impact-focused M\&E practices. The goal is to enhance goal alignment, communication, and the overall impact of joint initiatives.

In conclusion, this research offers new and practical perspectives for measuring and maximizing impact in complex collaborative settings. The study aims to illuminate pathways towards more effective and impactful multi-organizational collaborations.
\end{abstract}

\vspace{8pt}
\begin{IEEEkeywords}
Multi-Organizational Systems, Change Theory, Monitoring and Evaluation, M\&E, Impact-Driven Approach, Collaboration, Implementation Strategies, Organizational Ecosystems, Synergy, Communication in Multi-Organizational Settings, Alignment of Objectives, Shared Goals, Challenges in Implementation, Measuring Success, Lasting Change, Adaptability in Collaborations, Efficiency Metrics, Stakeholder Engagement, Inter-organizational Relationships, Framework for Impact Measurement, Best Practices in Multi-Organizational Collaborations.
\end{IEEEkeywords}
\vspace{16pt}

\section{Introduction}
\label{intro}
In a rapidly evolving global landscape, the need for effective multi-organizational systems has never been more apparent. This paper, inspired by the collaborative efforts within the UNITA project – an alliance of twelve European universities\cite{unita_unita_nodate}, seeks to develop an impact-focused Monitoring and Evaluation (M\&E) methodology. The UNITA project, with its commitment to inter-university cooperation and regional integration, serves as a prime example of the potential and challenges inherent in multi-organizational collaborations.

Drawing from the insights of the Methods Lab's 2016 paper, "When and How to Develop an Impact-Oriented M\&E system"\cite{hearn_when_2016}, this research aims to transcend traditional M\&E approaches. We focus on a broader perspective that emphasizes long-term impact and sustainable change, crucial for initiatives like UNITA.

The Theory of Change, is a structured way to understand the expected outcomes and impacts in complex collaborative environments. It is particularly relevant to the UNITA project, guiding our analysis on how inter-university collaborations could be optimized for greater effectiveness.

Through practical examples, including those from the UNITA initiative, this study illustrates the application of the Theory of Change in conjunction with impact-focused M\&E practices. We explore how these methodologies can improve goal alignment, communication, and the overall impact of joint initiatives.

The objective of this paper is to offer a critical examination of current M\&E methodologies in the context of multi-organizational systems like UNITA and to propose an enhanced framework that integrates the Theory of Change for more impactful evaluations.

In conclusion, this research contributes to the broader discourse on multi-organizational systems, providing new insights and practical approaches for measuring and maximizing impact. By focusing on projects like UNITA, we aim to shed light on the pathways to more effective and impactful multi-organizational collaborations.

\section{State of the Art}
\label{soa}


\subsection{Conclusions}

\section*{Acknowledgment}
\vspace{12pt}
\bibliographystyle{plain}
\bibliography{bib/references}

\end{document}
