\documentclass[conference]{IEEEtran}
\IEEEoverridecommandlockouts
% The preceding line is only needed to identify funding in the first footnote. If that is unneeded, please comment it out.
\usepackage{cite}
\usepackage{amsmath,amssymb,amsfonts}
\usepackage{algorithmic}
\usepackage{graphicx}
\usepackage{textcomp}
\usepackage{xcolor}
\def\BibTeX{{\rm B\kern-.05em{\sc i\kern-.025em b}\kern-.08em
    T\kern-.1667em\lower.7ex\hbox{E}\kern-.125emX}}
\begin{document}

\title{Implementing Multi-Organizational Systems: Strategies and Ecosystems through Change Theory and Impact-Driven M\&E\\
{\footnotesize \textsuperscript{}}
\thanks{}
}

\author{\IEEEauthorblockN{Clément Combier}
\IEEEauthorblockA{\textit{Master 2 SIGLIS} \\
\textit{Université de Pau et des Pays de l'Adour}\\
Anglet, France \\
clement.combier@etud.univ-pau.fr}}
\maketitle

\begin{abstract}
In today's interconnected world, the implementation of multi-organizational systems has become increasingly vital for collaborative success. This paper delves into the intricacies of establishing such systems, emphasizing the pivotal role of Change Theory and an impact-driven Monitoring and Evaluation (M\&E) approach. By exploring various strategies and ecosystems, we provide a comprehensive guide on how organizations can synergize their efforts, ensuring effective communication, alignment of objectives, and the realization of shared goals. Through a detailed analysis, we highlight the challenges faced during implementation and propose solutions rooted in Change Theory. Furthermore, we advocate for an M\&E approach that prioritizes impact, offering a framework that organizations can adapt to measure success not just in terms of outputs, but in terms of lasting change. This research serves as a beacon for institutions aiming to navigate the complexities of multi-organizational collaborations, ensuring efficiency, adaptability, and meaningful impact.
\end{abstract}

\vspace{8pt}
\begin{IEEEkeywords}
Multi-Organizational Systems, Change Theory, Monitoring and Evaluation (M\&E), Impact-Driven Approach, Collaboration, Implementation Strategies, Organizational Ecosystems, Synergy, Communication in Multi-Organizational Settings, Alignment of Objectives, Shared Goals, Challenges in Implementation, Measuring Success, Lasting Change, Adaptability in Collaborations, Efficiency Metrics, Stakeholder Engagement, Inter-organizational Relationships, Framework for Impact Measurement, Best Practices in Multi-Organizational Collaborations.
\end{IEEEkeywords}
\vspace{16pt}

\section{Introduction}


\section*{Acknowledgment}
\vspace{12pt}
\bibliographystyle{plain}
\bibliography{references}

\end{document}
